\newpage
\begin{center}
Studies in Logic\\
Logic and Argumentation\\
Volume ?
\vspace{10cm}

Reason to Dissent\\
Proceedings of the 3rd European Conference on Argumentation\\
Volume I
\end{center}

\newpage

\newpage

\begin{center}
Reason to Dissent\\
Proceedings of the 3rd European Conference on Argumentation\\
Volume II
\vspace{10cm}

Edited by\\
Catarina Dutilh Novaes, Henrike Jansen, Jan Albert van Laar and Bart Verheij
\end{center}

\newpage

Copyright things?

\newpage

\section*{Conference Organization}
	\addcontentsline{toc}{chapter}{Conference Organization}
	\subsection*{Organising committee}
	\begin{compactitem}[]
		\item[] Jan Albert Van Laar, chair (University of Groningen)
		\item[] Henrike Jansen (University of Leiden)
		\item[] Catarina Dutilh Novaes (Vrije Universiteit Amsterdam)
		\item[] Bart Verheij (University of Groningen)
	\end{compactitem}

	\subsection*{Programme committee}
	\begin{compactitem}[]
		\item[] Dale Hample, chair (University of Maryland)
		\item[] Henrike Jansen, secretary (University of Leiden)
		\item[] Marcin Lewinski (Universidade Nova de Lisboa)
		\item[] Frank Zenker (Lund University)
		\item[] Juho Ritola (University of Turku)
 	\end{compactitem}

 	\subsection*{ECA steering committee}
 	\begin{compactitem}[]
		\item[] Fabio Paglieri, chair (ISTC-CNR)
		\item[] Jan Albert Van Laar (deputy chair) (University of Groningen)
		\item[] Lilian Bermejo Luque (University of Granada)
		\item[] Katarzyna Budzyńska (Polish Academy of Sciences)
		\item[] Henrike Jansen (University of Leiden)
		\item[] Marcin Koszowy (University of Białystok)
		\item[] Marcin Lewiński (Universidade Nova de Lisboa)
		\item[] Dima Mohammed (Universidade Nova de Lisboa)
		\item[] Steve Oswald (University of Fribourg)
		\item[] Juho Ritola (University of Turku)
		\item[]Sara Rubinelli (University of Lucerne)
		\item[] Frank Zenker (University of Lund)
	\end{compactitem}


	\subsection*{Scientific Panel}
	\begin{compactitem}[]
		\item[] Adam Richards (Texas Christian University)
		\item[] Alessandra Von Burg (Wake Forest University)
		\item[] Amy Johnson (University of Oklahoma)
		\item[] Andrea Rocci (Università della Svizzera Italiana)
		\item[] Andrei Moldovan (University of Salamanca)
		\item[] Andrew Aberdein (Florida Institute of Technology)
		\item[] Assimakis Tseronis (Örebro University)
		\item[] Bart Garssen (University of Amsterdam)
		\item[] Bart van Klink (Vrije Universiteit Amsterdam)
		\item[] Beth Innocenti (University of Kansas)
		\item[] Carly Woods (University of Maryland)
		\item[] Catherine Hundleby (University of Windsor)
		\item[] Christoph Lumer (University of Siena)
		\item[] Christopher Tindale (University of Windsor)
		\item[] Constanza Ihnen (University of Chile, Law Faculty, Institute of Argumentation)
		\item[] Corina Andone (University of Amsterdam)
		\item[] Daniel Cohen (Colby College)
		\item[] David Godden (Michigan State University)
		\item[] David Hitchcock (McMaster University)
		\item[] David Williams (Florida Atlantic University)
		\item[] David Zarefsky (Northwestern University)
		\item[] Eveline Feteris (University of Amsterdam)
		\item[] Fabrizio Macagno (Universidade NOVA de Lisboa)
		\item[] Floriana Grasso (University of Liverpool)
		\item[] Francisca Snoeck Henkemans (University of Amsterdam)
		\item[] C. Goddu (University of Richmond)
		\item[] Gabor Zemplen (Budapest University of Technology and Economics)
		\item[] Gabrijela Kisicek (University of Zagreb)
		\item[] Gilbert Plumer (Law School Admission Council) (retired)
		\item[] Gregor Betz (KIT Karlsruher)
		\item[] Hans Hoeken (Utrecht University)
		\item[] Hans Vilhelm Hansen (University of Windsor)
		\item[] Harald Wohlrapp (Universität Hamburg)
		\item[] Harry Weger (University of Central Florida)
		\item[] Harvey Siegel (University of Miami)
		\item[] Hen Jans (University of Leiden)
		\item[] Henry Prakken (University of Groningen)
		\item[] Igor Z. Zagar (Educational Research Institute)
		\item[] Ingeborg van der Geest (University of Amsterdam)
		\item[] Isabela Fairclough (University of Cantral Lancashire)
		\item[] Anthony Blair (University of Windsor)
		\item[] Jacky Visser (University of Dundee)
		\item[] Jean Goodwin (North Carolina State University)
		\item[] Jérôme Jacquin (University of Lausanne)
		\item[] Joana Garmendia (University of the Basque Country)
		\item[] John Casey (Northeastern Illinois University)
		\item[] Jos Hornikx (Radboud University)
		\item[] Juho Ritola (University of Helsinki)
		\item[] Katharina Stevens (University of Lethbridge)
		\item[] Laura Vincze (Universityt of Pisa)
		\item[] Lena Wahlberg (Lund University)
		\item[] Leo Groarke (Trent University)
		\item[] Lotte Van Poppel (University of Amsterdam)
		\item[] Louis De Saussure (University of Neuchatel)
		\item[] Maarten van Leeuwen (Leiden University Centre for Linguistics) (LUCL)
		\item[] Manfred Kienpointner (University of Innsbruck)
		\item[] Manfred Kraus (University of Tübingen)
		\item[] Manfred Stede (University of Potsdam)
		\item[] Marianne Doury (Paris Descartes University)
		\item[] Mariusz Urbanski (Adam Mickiewicz University)
		\item[] Mark Aakhus (Rutgers University)
		\item[] Marta Zampa (Zurich University of Applied Sciences)
		\item[] Martin Hinton (University of Lodz)
		\item[] Martin Reisigl (University of Bern)
		\item[] Mehmet Ali Uzelgun (Universidade NOVA de Lisboa)
		\item[] Michael Baumtrog (Ryerson University)
		\item[] Michael Hoffmann (Georgia Institute of Technology)
		\item[] Michael Hoppmann (Northeastern University)
		\item[] Michel Dufour (Université Sorbonne-Nouvelle)
		\item[] Minghui Xiong (Sun Yat-sen University)
		\item[] Niki Pfeifer (Munich Center for Mathematical Philosophy)
		\item[] Patrick Bondy (Cornell University)
		\item[] Paula Olmos (Universidad Autónoma de Madrid)
		\item[] Rudi Palmieri (University of Liverpool)
		\item[] Ruth Amossy (Tel-Aviv University)
		\item[] Sara Greco (Università della Svizzera Italiana)
		\item[] Sarah Bigi (Università Cattolica del Sacro Cuore)
		\item[] Scott Aikin (Vanderbilt University)
		\item[] Steven Patterson (Marygrove College)
		\item[] Susan Kline (The Ohio State University)
		\item[] Thierry Herman (Universities of Neuchatel and lausanne)
		\item[] Yun Xie (Sun Yat-sen University)
	\end{compactitem}

\newpage

\section*{Keynote Speakers}
	\addcontentsline{toc}{chapter}{Keynote Speakers}
\subsection*{Critical thinking as discourse}
\emph{Deanna Kuhn}
\\\\
Less than it is an individual ability or skill, critical thinking is a dialogic practice people engage in and commit to, initially interactively and then in interiorized form with the other only implicit. An argument depends for its meaning on how others respond (Gergen, 2015). In advancing arguments, well-practiced thinkers anticipate their defeasibility as a consequence of others’ objections, in addition envisioning their own potential rebuttals. Whether in external or interiorized form, the dialogic process creates something new, while itself undergoing development.

This perspective may be useful in sharpening definition of the construct of critical thinking and in so doing help to bring together the largely separate strands of work examining it as a theoretical construct, a measurable skill, and an educational objective. Implications for education follow. How might critical thinking as a shared practice be engaged in within educational settings in ways that will best support its development? One step is to privilege frequent practice of direct peer-to-peer discourse. A second is to take advantage of the leveraging power of dialog as a bridge to individual argument – one affording students’ argumentative writing a well-envisioned audience and purpose. Illustrations of this bridging power are presented.   Finally, implications for assessment of critical thinking are noted and a case made for the value of people’s committing to a high standard of critical thinking as a shared and interactive practice.


\subsection*{Revisiting Apologie de la polémique: about some “felicity conditions” allowing for coexistence in dissent}
\emph{Ruth Amossy}
\\\\
In my book entitled Apologie de la polémique (2014), I claimed that polemical discourse fulfils various social functions, among which “coexistence in dissensus” seems the most important. It means not only that disagreement is the basis of life in society, and the principle on which argumentation as a common, rational search for the reasonable, is built. It also signifies that agreement cannot always be reached in democratic societies recognizing the importance of diversity and difference, so that disagreement has to be managed through verbal confrontations, namely, agonistic discussions and polemical exchanges. It thus appears that the latter, though generally blamed for its radicalization and polarization, plays an important role in the public sphere. Among others, public polemics helps opposite parties to voice conflicting opinions and fight for antagonistic solutions without recurring to arms. To use Chantal Mouffe’s words, it transforms “enemies” to be destroyed into “adversaries” who have a right to speak. Beside other social functions discussed in the book, polemics authorizes what the French call a “vivre-ensemble” – the possibility for people who do not share the same opinions, if not the same premises, to share the same national space and live together without outbursts of violence.

However, the emphasis on dissent and its polemical management is not without raising multiple questions concerning the conditions of possibility and the limits of the so-called coexistence in dissent. Obviously, the use of polemical discourse is not enough to prevent citizens from physically fighting each other and even, sometimes, to dispel the specter of civil war. Outbursts of violence against refugees regularly occur in Germany where the polemical discussion is vivid. In France, the polemical exchanges on Emmanuel Macrons’ reforms and the authorized street demonstrations did not prevent urban violence. Even if polemical campaign discourse is tolerated, it did not prevent armed confrontations in certain African countries such as Ivory Coast. What, then, are the “felicity conditions” needed in order for public polemics to secure a peaceful “living together” in the framework of persistent and sometimes deep disagreements that can hardly be avoided in the democratic space? My contention is that to answer this question, it is necessary to explore polemical confrontations in their institutional framework, and to examine the functioning of polemical discourse in relation to the political, forensic and cultural factors that determine (at least partly) its degree of success. After synthetizing the finding of my first research into dissent and its polemical management, I will try – on the basis of a few case studies – to gather some of the “felicity conditions” necessary to make coexistence in dissent possible.

\subsection*{Dissent needed: argumentation for AI and law applications}
\emph{Katie Atkinson}
\\\\
As technological advances in artificial intelligence are being turned into deployed products, societal questions are being raised about the need for AI tools to be able to explain their decisions to humans.  This need becomes even more pressing when AI technologies are applied in domains where critical decisions are made that can result in a significant effect upon individuals or groups in society.  One such domain is law, where there is a thriving market developing in support tools for assisting with a variety of legal tasks carried out within law firms and the wider legal sector.  Law is a domain rich in argumentation and support tools that are used to aid legal decision making should similarly be able to explain why a particular outcome of a decision has been reached, and not an alternative outcome.  Dissent needs to be captured and revealed within AI reasoners to ensure that the decision space is explored from different perspectives, if AI tools are to be deployed effectively to assist with legal reasoning tasks.  In this talk I will discuss a body of work on computational models of argument for legal reasoning and show how dissent features within this work to promote scrutability of AI decision making. 
