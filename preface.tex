\section*{Preface}
\addcontentsline{toc}{chapter}{Preface}

BNAIC is the annual Benelux Conference on Artificial Intelligence. In 2017, the 29\textsuperscript{th} edition of BNAIC is organized by the Institute of Artificial Intelligence and Cognitive Engineering (ALICE), University of Groningen, under the auspices of the Benelux Association for Artificial Intelligence (BNVKI) and the Dutch Research School for Information and Knowledge Systems (SIKS).

BNAIC 2017 takes place in Het Kasteel, Melkweg 1, Groningen, The Netherlands, on Wednesday November 8 and Thursday November 9, 2017. BNAIC 2017 includes invited speakers, research presentations, posters, demonstrations, a deep learning workshop (organized by our sponsor NVIDIA) and a research and business session. 

The four BNAIC 2017 keynote speakers are:

\begin{itemize}
	\item 
Marco Dorigo, Universit\'{e} Libre de Bruxelles\\
\emph{Swarm Robotics: Current Research Directions at IRIDIA}

	\item 
Laurens van der Maaten, Facebook AI Research\\
\emph{From Visual Recognition to Visual Understanding}

	\item 
Luc Steels, Institute for Advanced Studies (ICREA), Barcelona\\
\emph{Digital Replicants and Mind-Uploading
}

	\item 
Rineke Verbrugge, University of Groningen\\
\emph{Recursive Theory of Mind: Between Logic and Cognition}
\end{itemize}

\noindent Three FACt talks (FACulty focusing on the FACts of Artificial Intelligence) are scheduled:

\begin{itemize}
	\item 
Bert Bredeweg, Universiteit van Amsterdam\\
\emph{Humanly AI: Creating smart people with AI}
	\item Eric Postma, Tilburg University\\
\emph{Towards Artificial Human-like Intelligence}
	\item Geraint Wiggins, Queen Mary University of London/Vrije Universiteit Brussel\\
\emph{Introducing Computational Creativity}
\end{itemize}

\noindent Authors were invited to submit papers on all aspects of Artificial Intelligence. This year we have received 68 submissions in total. Of the 30 submitted Type A regular papers, 11 (37\%) were accepted for oral presentation, and 14 (47\%) for poster presentation. 5 (17\%) were rejected. Of the 19 submitted Type B compressed contributions, 17 (89\%) were accepted for oral presentation, and 2 (11\%) for poster presentation. None were rejected. All 6 submitted Type C demonstration abstracts were accepted. Of the submitted 13 Type D thesis abstracts, 5 (38\%) were accepted for oral presentation, and 8 (62\%) for poster presentation. None were rejected. The selection was made using peer review. Each submission was assigned to three members of the program committee, and their expert reviews were the basis for our decisions. 

All submissions accepted for oral or poster presentations and all demonstration abstracts appear in the electronic preproceedings, made available on the conference web site during the conference (\url{http://bnaic2017.ai.rug.nl/}). All 11 Type A regular papers accepted for oral presentation will appear in the postproceedings, to be published in the Springer CCIS series after the conference.

The BNAIC 2017 conference would not be possible without the support and efforts of many. We thank the members of the program committee for their constructive and scholarly reviews. We are grateful to 
    Elina Sietsema, Carlijne de Vries and 
		Sarah van Wouwe, members of the administrative staff at the Institute of Artificial Intelligence and Cognitive Engineering (ALICE), for their tireless and reliable support. We thank our local organisation team 
		Luca Bandelli,
    Abe Brandsma,
    Tomasz Darmetko,
    Mingcheng Ding,
    Ana Dugeniuc,
    Joel During,
    Ameer Islam,
    Siebert Looije,
    Ren\'{e} Mellema,
    Michaela Mr\'{a}zkov\'{a},
    Annet Onnes,
    Benjamin Shaffrey,
    Sjaak ten Caat,
    Albert Thie,
    Jelmer van der Linde,
    Luuk van Keeken,
    Paul Veldhuyzen,
    Randy Wind, and
    Galiya Yeshmagambetova, 
all students in our BSc and MSc Artificial Intelligence programs, for enthusiastically volunteering to help out in many ways. We thank Annet Onnes for preparing the preproceedings, Jelmer van der Linde for developing the web site, Randy Wind for designing the program leaflet, and Albert Thie for coordinating the local organisation.
		
We are grateful to our sponsors for their generous support of the conference:
\begin{itemize}
%		
	\item 
Target Holding
	\item 
NVIDIA Deep Learning Institute
%
	\item 
Anchormen
	\item 
Quint
	\item 
the Netherlands Research School for Information and Knowledge Systems (SIKS)
	\item 
SIM-CI
%
	\item 
Textkernel
	\item 
LuxAI
	\item 
IOS Press
	\item 
Stichting Knowledge-Based Systems (SKBS)
	\item 
SSN Adaptive Intelligence 
\end{itemize}

\noindent 
We wish you a pleasant conference!

\begin{flushright}
Bart Verheij \& Marco Wiering
\end{flushright}