\section*{Preface}
\addcontentsline{toc}{chapter}{Preface}
\pdfbookmark[chapter]{Preface}{preface}

After two successful editions held in Lisbon in 2015 and Fribourg in 2017, ECA was hosted in 2019 by the Faculty of Philosophy of the University of Groningen, on 24-27 June 2019. These two volumes contain the Proceedings of this third edition of the conference series, whose special theme was Reason to Dissent.

The European Conference on Argumentation (ECA) is a pan-European biennial initiative aiming to consolidate and advance various strands of research on argumentation and reasoning by gathering scholars from a range of disciplines. While based in Europe, ECA involves and encourages participation by argumentation scholars from all over the world; it welcomes submissions linked to argumentation studies in general, in addition to those tackling the conference theme. The 2019 Groningen edition focused on dissent. The goal was to inquire into the virtues and vices of dissent, criticism, disagreement, objections, and controversy in light of legitimizing policy decisions, justifying beliefs, proving theorems, defending standpoints, or strengthening informed consent. It is well known that dissent may hinder the cooperation and reciprocity required for reason-based deliberation and decision-making. But then again, dissent also produces the kind of scrutiny and criticism required for reliable and robust outcomes. How much dissent does an argumentative practice require? What kinds of dissent should we promote, or discourage? How to deal with dissent virtuously? How to exploit dissent in artificial arguers? How has dissent been conceptualized in the history of rhetoric, dialectic and logic? The papers in these two volumes discuss these and other questions pertaining to argumentation and dissent (among other themes).

The conference had 224 participants, and 188 papers were presented. These high numbers are a clear indication that ECA continues to fulfill its role as a key platform of scholarly exchange in the field of argumentation. The contents of these two volumes can be regarded as a reflection of the current state of the art in argumentation scholarship in general.

The proceedings contain papers that were accepted based on abstract submissions; each submission was thoroughly evaluated by three reviewers of our scientific board—for a full list of ECA committees, see www.ecargument.org. Volume I gathers ?? long papers and associated commentaries, together with ?? papers presented in the ?? thematic panels that were held during ECA2019. Volume II gathers ?? regular papers that were presented during the conference. 

Many people have contributed to the success of ECA 2019, and for the completion of the Proceedings. First of all, we must thank all members of our Scientific Panel and of our Programme Committee, thanks to whom we were able to select papers of the highest quality. In Groningen, thanks to those who provided organizational support, in particular the team of student assistants (especially Johan Rodenburg) who ensured that the conference was a pleasant experience to all participants. Our heartfelt thanks go to Jelmer van der Linde and Annet Onnes, who accomplished the gigantic task of putting all the papers together into these two volumes, and assisted us throughout in the process of producing the Proceedings.

The next edition of ECA will take place in Rome in 2021, and we look forward to seeing the ECA community gathering again for another successful event.

\medskip

\noindent Catarina Dutilh Novaes, Henrike Jansen, Jan Albert van Laar, Bart Verheij